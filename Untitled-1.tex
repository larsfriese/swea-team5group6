\newpage

\section{Hermon Gimikael}
\subsection{Requirement 1 : User Challenge}

\begin{table}[h!]
    \begin{tabularx}{\textwidth}{|>{\raggedright\arraybackslash}p{0.25\textwidth}|X|}
        \hline
        Name             & Allow user to create and join fitness Challenges and competitions                               \\ \hline
        ID               & 1                                                                                   \\ \hline
        Business Value   & Medium                                                                                  \\ \hline
        Description      & Allows the user to create a fitness Challenge for himself and to join or invite other logged in user\\ \hline
        Trigger          & User pushs create and share button or pushs join button \\ \hline
        Actors           & User1,User2,User Challenge System                        \\ \hline
        Pre-conditions   & User1 and User2 successfully logged in, internet connection                                  \\ \hline
        Post-conditions  & User can participate a Challenge alone or with User2                                                        \\ \hline
        Basic Flow       & This is the main scenario were User1 creates a Challenge and Invite User2 who accept it\\ \hline
                         & Actions: \\
                         & 1. User1 set sports category,period and goal\\
                         & 2. User1 pushs "share" button \\
                         & 3. User1 chooses User2 to compete with\\
                         & 4. User Challenge System sends invitation to User2 \\
                         & 5. User2 receives invitation\\
                         & 6. User Challenge System generates same Challenge page for User1 and User2 \\ \hline
        Alternative Flow A & User2 rejects the invitation\\
                         & Actions: \\
                         & 1. System sends User1 notification that User2 rejected his invitation \\
                         & 2. User1 pushes share button \\
                         & 3. User1 chooses User3 to compete with\\
                         & 4. System sends invitation to User3 \\
                         & 5. User3 pushes"accept" button \\
                         & 6. Challenge System generates same Challenge page for User1 and User3 \\ \hline
        Alternative Flow B & User1 sent wrong User invitation \\
                         & Actions: \\
                         & 1. User1 push "show Invitation list" \\
                         & 2. User1 pushs "cancel" button \\
                         & 3. System withdraw invitation notification from User2\\ \hline
    \end{tabularx}
\end{table}

\paragraph{Functional requirements}
		\begin{itemize}
			\item  User should be able to create new fitness challenges by specific details like burned calories or
			challenge names
			\item  Users have the possibility to join already existing fitness challenges or competitions
		\end{itemize}
		
	\paragraph{Non-functional requirements}
		\begin{itemize}
			\item The System has to be capable to handle multiple users participating in the same challenges or
			competitions at the same time without losing performance quality
			
		\end{itemize}

\clearpage

\begin{figure}[htbp]
	\textbf{Hermon Gimikael}
    \centering
	\begin{subfigure}{\textwidth}
		\resizebox{\textwidth}{!}{\includesvg[]{Hermon/UserChallenge_UseCase.svg}}
		\caption{Use Case Diagram}
	\end{subfigure}
\end{figure}

\clearpage

\begin{figure}[htbp]
	\textbf{Hermon Gimikael}
	\centering
	\begin{subfigure}{\textwidth}
		\resizebox{\textwidth}{!}{\includesvg[]{Hermon/UserChallenge_Activity.svg}}
		\caption{Activity Diagram}
	\end{subfigure}
\end{figure}

\clearpage

\begin{figure}[htbp]
	\textbf{Hermon Gimikael}
	\centering
	\begin{subfigure}{\textwidth}
		\resizebox{\textwidth}{!}{\includesvg[]{Hermon/UserChallenge_ClassDiagram.svg}}
		\caption{Class Diagram}
	\end{subfigure}
\end{figure}

\clearpage

\begin{figure}[htbp]
	\textbf{Hermon Gimikael}
	\centering
	\begin{subfigure}{\textwidth}
		\resizebox{\textwidth}{!}{\includesvg[]{Hermon/UserChallenge_SequenceDiagram.svg}}
		\caption{Sequence Diagram}
	\end{subfigure}
\end{figure}


\clearpage

\textbf{Hermon Gimikael}
\subsection{Requirement 2 : Medication Reminder System}


\begin{table}[h!]
    \begin{tabularx}{\textwidth}{|>{\raggedright\arraybackslash}p{0.25\textwidth}|X|}
        \hline
        Name             & Medication Reminder System                               \\ \hline
        ID               & 2                                                                                     \\ \hline
        Business Value   & low                                                                                  \\ \hline
        Description      & Service to help user adher to their medication schedules  \\ \hline
        Trigger          & User adds medication and sets his Medication Details\\ \hline
        Actors           & User,Medication Reminder System                               \\ \hline
        Pre-conditions   & User sat his Medication and Details                                    \\ \hline
        Post-conditions  & User took his medication on time                                                   \\ \hline
        Basic Flow       & This is the main scenario when the user sat his Medication and Details \\ \hline
                         & Actions: \\
                         & 1.User sets his name, age, weight, medication, dosis, notification time, urgency Level(low,medium,high) and emergency contact\\
                         & 2. System sends a push notification at the sat time to remind the user to take his Medicine \\
                         & 3. After Reminder System sends a second notification that user should confirm or deny medication Intake\\
                         & 4. User confirms \\
                         & 5. System logs confirmation \\ \hline
        Alternative Flow A & User denies or doesn't respond to confirmation notification \\
                         & Actions: \\
                         & 1. User denies or doesn't respond \\
                         & 2. System checks medication urgency Level \\
                         & 3. If "low" or "medium" System sends hourly push notifications as reminder 
                           to confirm or denie \\
                         & 4. If "high" System sends notification to emergency contact that the User didn't took his Medicin \\ \hline
        Alternative Flow B & User want to change his Medication Details \\
                         & Actions: \\
                         & 1. User clicks "Details" button \\
                         & 2. User clicks "adjust" button \\
                         & 3. User clicks "Change" button \\
                         & 4. User changes the Deatails \\
                         & 5. System changes the Medication Details\\ \hline
    \end{tabularx}
\end{table}

\paragraph{Functional requirements}
		\begin{itemize}
			\item  The system should be able to remind users promptly of upcoming medication doses based on schedules set by the user.
			\item  Users should be able to customize reminders according to their individual needs, including frequency of notifications and the ability to add or remove specific medications
		\end{itemize}
		
	\paragraph{Non-functional requirements}
		\begin{itemize}
			\item The system must be reliable, delivering reminders at the correct times to ensure users take their medications on time.
			\item : The user interface should be intuitive and user-friendly, facilitating easy interaction and configuration for users of all ages.
		\end{itemize}

\clearpage

\begin{figure}[htbp]
	\textbf{Hermon Gimikael}
	\centering
	\begin{subfigure}{\textwidth}
		\resizebox{\textwidth}{!}{\includesvg[]{Hermon/MedicationReminder_UseCase.svg}}
		\caption{Use Case Diagram}
	\end{subfigure}
\end{figure}

\clearpage


\begin{figure}[htbp]
	\textbf{Hermon Gimikael}
	\centering
	\begin{subfigure}{\textwidth}
		\resizebox{\textwidth}{!}{\includesvg[]{Hermon/MedicationReminder_ActivityDiagram.svg}}
		\caption{Activity Diagram}
	\end{subfigure}
\end{figure}

\clearpage

\begin{figure}[htbp]
	\textbf{Hermon Gimikael}
	\centering
	\begin{subfigure}{\textwidth}
		\resizebox{\textwidth}{!}{\includesvg[]{Hermon/MedicationReminder_ClassDiagram.svg}}
		\caption{Class Diagram}
	\end{subfigure}
\end{figure}

\clearpage

\begin{figure}[htbp]
	\textbf{Hermon Gimikael}
	\centering
	\begin{subfigure}{\textwidth}
		\resizebox{\textwidth}{!}{\includesvg[]{Hermon/MedicationReminder_SequenceDiagram.svg}}
		\caption{Sequence Diagram}
	\end{subfigure}
\end{figure}

\clearpage

\textbf{Hermon Gimikael}
\subsection{Requirement 3 : Water Intake Service}

\begin{table}[h!]
    \begin{tabularx}{\textwidth}{|>{\raggedright\arraybackslash}p{0.25\textwidth}|X|}
        \hline
        Name             & Water Intake Service                                \\ \hline
        ID               & 3                                                                                     \\ \hline
        Business Value   & High                                                                                    \\ \hline
        Description      & Gives option to monitor and manage daily water consumption \\ \hline
        Trigger          & User adds water intake  \\ \hline
        Actors           & User,Water Intake Service                                \\ \hline
        Pre-conditions   & Water Intake Service opened                                 \\ \hline
        Post-conditions  & Water Intake Service has recorded and updated users Water consumption                                                     \\ \hline
        Basic Flow       & This is the main scenario when the user added water throught the day and reached his goal \\ \hline
                         & Actions: \\
                         & 1. User sets his sex,age,weight\\
                         & 2. System generated default Value \\
                         & 3. User accepts Value \\
                         & 4. System saves Value \\
                         & 5. User adds water intake throught the day\\
                         & 6. After one Time period system calculates total Intake \\
                         & 7. User reached his Goal \\
                         & 8. System notifies user about reached goal and logs achievment of the day\\ 
                         & 9. System resets water intake to 0\\ \hline
        Alternative Flow A & User missed Water intakes  \\
                         & Actions: \\
                         & 1. Syste informs user about missed goal\\
                         & 2. System gives user the possibility to enter missed Water intakes \\ 
                         & 3. user enters remaining Water\\
                         & 4. System calculates new total Intake\\
                         & 5. User reached his Goal\\ 
                         & 6. System notifies user about reached goal and logs achievment of the day\\
                         & 7. System resets water intake to 0 \\ \hline
        Alternative Flow B & User didn't reach his goal\\
                         & Actions: \\
                         & 1. System informs user about missed goal \\
                         & 2. User declines possibility to enter missed Water intakes\\
                         & 3. System informs user about missed goal and sends reccomandations for next time \\ \hline
    \end{tabularx}
\end{table}
	
\paragraph{Functional requirements}
		\begin{itemize}
			\item  The system should allow users to track their daily water intake, including the quantity of water consumed and the time of consumption.
			\item  Users should be able to set personalized water intake goals based on factors such as age, weight, and activity level.
		\end{itemize}
		
	\paragraph{Non-functional requirements}
		\begin{itemize}
			\item The system must reliably record and display water intake data to accurately track user progress.
			\item The user interface should be intuitive and easy to navigate, allowing users to input water intake data quickly and efficiently.
		\end{itemize}


\clearpage

\begin{figure}[htbp]
    \textbf{Hermon Gimikael}
    \centering
    \begin{subfigure}{\textwidth}
        \resizebox{\textwidth}{!}{\includesvg[]{Hermon/WaterIntake_UseCase.svg}}
        \caption{Use Case Diagram}
    \end{subfigure}
\end{figure}

\clearpage

\begin{figure}[htbp]
    \textbf{Hermon Gimikael}
    \centering
    \begin{subfigure}{\textwidth}
        \resizebox{\textwidth}{!}{\includesvg[]{Hermon/WaterIntake_ActivityDiagram.svg}}
		\caption{Activity Diagram}
	\end{subfigure}
\end{figure}

\clearpage

\begin{figure}[htbp]
    \textbf{Hermon Gimikael}
    \centering
    \begin{subfigure}{\textwidth}
        \includesvg[width=\textwidth]{Hermon/WaterIntake_ClassDiagram.svg}
        \caption{Class Diagram}
    \end{subfigure}
\end{figure}

\clearpage

\begin{figure}[htbp]
    \textbf{Hermon Gimikael}
    \centering
    \begin{subfigure}{\textwidth}
        \includesvg[width=\textwidth]{Hermon/WaterIntake_SequenceDiagram.svg}
        \caption{Sequence Diagram}
    \end{subfigure}
\end{figure}

\clearpage

\textbf{Hermon Gimikael}
\subsection{Requirement 4 : Mood Level Tracker}

\begin{table}[h!]
    \begin{tabularx}{\textwidth}{|>{\raggedright\arraybackslash}p{0.25\textwidth}|X|}
        \hline
        Name             & Mood Level Tracker                               \\ \hline
        ID               & 4                                                                                      \\ \hline
        Business Value   & High                                                                                   \\ \hline
        Description      & User can track his Mood for good Mental Health beeing  \\ \hline
        Trigger          & User chooses Mood of the day and pushs "entry" button \\ \hline
        Actors           & User, Mood Tracker System                               \\ \hline
        Pre-conditions   & Mood Tracker System opened                                   \\ \hline
        Post-conditions  & User has an overlook about his well beeing                                                       \\ \hline
        Basic Flow       & This is the main scenario were the user entry his Mood of the day and his Mood is good or okay and want a Quick diary \\ \hline
                         & Actions: \\
                         & 1. User click Mood entry\\
                         & 2. System presents Mood scale \\
                         & 3. User selects Mood Level \\
                         & 4. User wants a Quick diary \\
                         & 5. System generates 10 quastions for user \\
                         & 6. Sysem logs mood entry with timestamp and quick entry\\ \hline
        Alternative Flow A & User selects "very sad " as mood entry \\
                         & Actions: \\
                         & 1. System recognize low mood Level \\
                         & 2. System suggests resources for mood improvement \\
                         & 3. User accept suggestion \\
                         & 4. System sends User links to articles and exercises \\
                         & 5. System loggs mood entry with timestamp and article links \\ \hline
        Alternative Flow B & User declines the Quick diary and summary  \\
                         & Actions: \\
                         & 1. System logs mood entry with timestamp \\
                         & 2. Systeme provides summary of recent mood entries\\
                         & 3. User declines summary \\
                         & 4. System closes \\ \hline
    \end{tabularx}
\end{table}

\paragraph{Functional requirements}
		\begin{itemize}
			\item  The system should allow users to log their mood levels on a daily basis, selecting from a predefined set of options
		\end{itemize}
		
	\paragraph{Non-functional requirements}
		\begin{itemize}
			\item The system must reliably record and display water intake data to accurately track user progress.
			\item The user interface should be intuitive and easy to navigate, allowing users to input water intake data quickly and efficiently.
		\end{itemize}

\clearpage

\begin{figure}[htbp]
    \textbf{Hermon Gimikael}
    \centering
    \begin{subfigure}{\textwidth}
        \resizebox{\textwidth}{!}{\includesvg[]{Hermon/MoodLevelTracker_UseCase.svg}}
        \caption{Use Case Diagram}
    \end{subfigure}
\end{figure}

\clearpage


\begin{figure}[htbp]
    \textbf{Hermon Gimikael}
    \centering
    \begin{subfigure}{\textwidth}
        \resizebox{\textwidth}{!}{\includesvg[]{Hermon/MoodLevelTracker_ActivityDiagram.svg}}
        \caption{Activity Diagram}
    \end{subfigure}
\end{figure}

\clearpage

\begin{figure}[htbp]
    \textbf{Hermon Gimikael}
    \centering
    \begin{subfigure}{\textwidth}
        \resizebox{\textwidth}{!}{\includesvg[]{Hermon/MoodTrackerSystem_ClassDiagram.svg}}
        \caption{Class Diagram}
    \end{subfigure}
\end{figure}

\clearpage

\begin{figure}[htbp]
	\textbf{Hermon Gimikael}
	\centering
	\begin{subfigure}{\textwidth}
		\resizebox{\textwidth}{!}{\includesvg[]{Hermon/UserChallenge_UseCase.svg}}
		\caption{Sequence Diagram}
	\end{subfigure}
\end{figure}


\clearpage